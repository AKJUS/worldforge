%\documentclass{article}
\documentclass{report}
\title{Atlas Semantic GeoMaps}
\author{Aloril}
\date{2002-02-21}
\usepackage{atlas}
\def\OOG{OOG}
\def\smile{;-)}

\begin{document}
   \maketitle
   \tableofcontents

   \newpage

   \section{Thanks}
   
   Some editing and comments by \contrib{Dan Tomalesky}{Grimicus}{grim@xynesis.com}
   and by \contrib{Bryce Harrington}{bryce}{bryce@neptune.net}
   Bryce also has written ``Discussion with bryce about containers'' -section\\

   Comments and suggestions for this whitepaper by (in order of time):\\
   \contrib{Thierry Mallard}{shaman}{thierry@mallard.com}
   \contrib{Adam Wendt}{adam}{billyjoeray@netzero.net}
   \contrib{Sami M\"akel\"a}{smkl}{sajuma@utu.fi}
   \contrib{James Best}{jamie}{jbest@magnacom.net}
   \contrib{Anubis}{anubis}{anubis@worldforge.org}
   \contrib{Rakshasa}{rakshasa}{rakshasa@start.no}
   \contrib{Miguel Guzm�n}{Aglanor}{aglanor@teleline.es}
   \contrib{Christian Reitwiesser}{KCN}{christian@reitwiessner.de}
   \contrib{James Turner}{James}{jmtn@blueyonder.co.uk}
   \contrib{Sal Ferro}{Sal}{sferro@wojo.com}
   \contrib{Jonathan Giszczak}{Dragon Master}{dragonm@assault.nexuslabs.com}
   \contrib{}{tbp}{tbp@tbp.dyndns.org}
   \contrib{Mike Taylor}{bear}{bear@code-bear.dyndns.org}
   \contrib{Malcolm Walker}{malcolm}{malcolm@worldforge.org}
   \contrib{Jorrit Tyberghein}{Jorrit}{Jorrit.Tyberghein@uz.kuleuven.ac.be}
\\

   Ideas are based on countless other people commenting at mail and
   IRC.

   Discussion logs:\\
   \link{http://http://brenda.worldforge.org:8080/logs/info.php3?file=forge200012300743.irc}\\
   \link{http://http://brenda.worldforge.org:8080/logs/info.php3?file=forge200012301148.irc}\\
   \link{http://http://brenda.worldforge.org:8080/logs/info.php3?file=forge200012301607.irc}\\
   \link{http://http://brenda.worldforge.org:8080/logs/info.php3?file=forge200101040947.irc}\\
   \link{http://http://brenda.worldforge.org:8080/logs/info.php3?file=forge200101111430.irc}\\
   \link{http://http://brenda.worldforge.org:8080/logs/info.php3?file=forge200101201220.irc}\\
   \link{http://brenda.worldforge.org:8080/logs/info.php?file=coders200201311235.irc}\\
   \link{http://brenda.worldforge.org:8080/logs/info.php?file=forge200202061527.irc}\\
   \link{http://brenda.worldforge.org:8080/logs/info.php?file=coders200202181345.irc}\\
   \link{http://brenda.worldforge.org:8080/logs/info.php?file=coders200202200025.irc}\\
   Comments by mail:\\
   \link{http://mail.worldforge.org/archive/protocols/2000-December/000110.html}

%   Home of this whitepaper:\\
%   \link{http://www.worldforge.net/aloril/atlas/whitepapers/semantic_maps.pdf}

   \newpage

   \section{Introduction}
   
   \subsection{Atlas}
   
   Atlas is the standard worldforge protocol for client/server interaction. For more
   about Atlas see \link{http://www.worldforge.org/dev/eng/protocols/}.
   
   In order to avoid confusing the concepts of a geographical map and how Atlas currently 
   defines a map, this paper will call the former {\em GeoMaps}.
   The purpose of paper is to raise discussion about what should be in the
   Atlas specification about geomaps and how it will relate to media.

   \subsection{Semantic GeoMaps}
   
   Semantic geomaps contain the information needed by a world server
   to model a game or other worlds. It provides a comprehensive
   listing of what objects are located in a given portion of the
   world, attributes of those objects, and their positions. Those
   attributes should give meaning to their objects. For example, a
   road object is used instead of line object ~(can this be
   elaborated?  is a road object a line object with special attributes
   or just a change of attribute?). Clients use an object class and
   its attributes to find the appropriate graphical media. Different
   clients will use different media (2D, 3D, text, aural, etc.) to
   represent the same object. This effectively separates the data,
   semantic geomaps, from the representation the client assigns it.

   In this way, Atlas semantic geomaps resemble object-oriented
   GIS\footnote{Geographic Information System
     \link{http://www.usgs.gov/research/gis/title.html}} systems (such
   as Smallworld or LAMPS) rather than the more traditional vector
   oriented GIS systems, such as MapInfo and ArcView. An Atlas geomap's primary
   difference from OO GIS is that it defines a certain standard set of object
   types and attributes, whereas in OOGIS, objects are constantly redefined from 
   geomap to geomap.  This is much like how in C++ anyone can create a class,
   but STL defines a standard set of classes for everyone to use, rather than
   creating their own personalized versions.  In addition, Atlas defines a 
   default set of values for the standard attributes for each type.  This 
   further allows a consistant way of defining GeoMaps.
   
   AI in NPCs and user scripts can use semantic information defined in these
   objects to follow a road or to decide how to react to other objects. 
   Also, they enable smarter tools than pure geographic info would
   allow.  Some example tools might:  find the shortest route between two
   places, select all roads, water, mountains, etc\ldots, 
   or apply erosion to river drainage basins.
   From an interface point of view, a two-dimensional, isometric client might 
   represent the game world's geomaps as a set of tiles laid out in a grid, 
   with animated sprites moving around on top, whereas the three-dimensional
   client might use a collection of textured meshes to describe the
   same geomaps.  A text client might provide only descriptions of the 
   entities in the game, in a MUD\footnote{MultiUser Dungeon 
   \link{http://www.mudconnect.com/mudfaq/mudfaq-p1.html}} style of layout. These 
   examples are not the limits.  A client could have multiple types 
   of media to display a specific geomap.

   Beyond this separation of media and object, identification--a trait of 
   semantic geomaps that distinguishes them from ordinary geomaps--allows 
   an object to contain more than just position and related geographical 
   information. ~(decide whether to make this example of attribute or what...)

   \section{Object model overview}

   Here we will attempt to explain the format for Atlas Objects.  There are five 
   basic types of attributes that an atlas object can have:
   \begin{description}
   \item[Integer] A number without decimal precision
   \item[Double] A number with decimal precision
   \item[String] A ordered collection of characters.  
   \item[List] An ordered collection of any altas type defined here. Members of  
               a list are not necessarily indexed, but can be.  This index is 
               typically an integer.
   \item[Map] An unordered collection of atlas types defined here, where each 
              member of the collection is indexed by a string (called a {\em key})
   \end{description}
   Altas objects are made up in various combinations of the above types.  The 
   following examples will further explain how Atlas defines objects and their
   classes.

   \subsection{Oak object example}

   This example shows a typical use of an Atlas Object.
   %using oak tree in 2 places
   \def\OakTree{
     \begin{Amap}
       \Attr{id}{``oak1234''}
       \Attr{name}{``The Giant Oldie''}
       \Attr{objtype}{``obj''}
       \Attr{parents}{(``oak'')}
       \Apos{12.5}{100.5}{0.5}
       \Avelocity{0.0}{0.0}{0.0}
       \Attr{loc}{``forest123''}
       \Attr{contains}{(``human234'')}
     \end{Amap}}
   \begin{Aobjectfigure}
     \OakTree
     \caption{Oak object example}
   \end{Aobjectfigure}
   \\
   Here's an explanation of the above attributes:
   \begin{description}
   \item[Id] -- The unique identifier for the object.
   \item[Name] -- The name of object. The name doesn't need to be unique
               in the world.
   \item[Objtype] -- The kind of Atlas object.
                  An ``obj'' is an instance of its parent(see below).
                  These types can specify ``in game'' or ``out of game'' objects.
                  Ex. Character object(in) v. Player object(out)
                  Other possible Objtype values are ``op'', ``class'', or
                  ``op\_definition''.  ``op'' is the same as ``obj'' except
                  that it pertains to an operation instance as opposed to object 
                  instance.  ``class'' and ``op\_definition'' will be explained
                  below.
   \item[Parents] -- The inheritence structure of the object.
                  Objects inherit attributes from a parent class. Classes may
                  inherit attributes from one or many parent classes. In the case
                  of an instance object, its parent is its class object's id 
                  (``oak'' in this example).
   \item[Pos] -- The position of object in relation to its container
              Usually the position is given as list of x, y and z.
              Direction of x, y and z are mathematical ones: x is from
              left to right or west to east, y is south to north and z is
              from down to up.
              \footnote{Worlds with more than three dimensions are possible,
              but are rare. Also, some games servers might want to use 
              different method to specify position.  For example sim games set
              in the Earth/Moon space, where the position is specified by six
              parameters, may want to use different way to specify position.
              In this case server should by default use the $pos$ attribute.
              If the client supports an alternative system, then the server
              can use that instead: for example $orbit\_pos$.}
   \item[Velocity] -- The velocity vector of object in relation to
                   the container.
                   \footnote{An object's pos and velocity should be in reference
                   to their container.}
   \item[Loc] -- The id of the object that contains this object. The object's container.
   \item[Contains] -- An id list specifying which objects are contained by this one.
   \end{description}
   
   \subsection{Oak class definition example}

   This is the definition of the oak class that the above object is an instance of:
   \label{example:oakclass}
   \begin{Aobjectfigure}
     \begin{Amap}
       \Attr{id}{``oak''}
       \Attr{name}{``Quercus''}
       \Attr{objtype}{``class''}
       \Attr{parents}{(``tree'')}
     \end{Amap}
     \caption{Oak class definition example}
   \end{Aobjectfigure}
   
   There are no new attributes in the oak class definition example
   compared to oak trees instances in the world.  Despite this, The descriptions of
   each are some what different:
   \begin{description}
   \item[Id] is the unique class name.
   \item[Name] This field is typically just a placeholder without real significance 
               to the class.  In this case, the Latin name is used. Usually, it is
               identical to id attribute. It was made intentionally different
               here to show it is possible \smile.
   \item[Objtype] The value ``class'' specifies a new object type. These objects
                  are out of game.  The purpose of these definitions is to provide
                  the layout of the class for any instances, and also to store default
                  values for the object type.  A benefit of this might be to allow the
                  server to pass any new definitions to the client via atlas. 
                  (if that functionality is implemented). ``op\_definition'' 
                  defines a new operation type.  It is basically the class definition 
                  for operations--much like ``class''.
   \item[Parents] All superclasses of this class definition are placed in this list.
   \end{description}
   
   \section{Containers}
   %\begin{enumerate}
   Containers are the organization of objects in geomaps.  Each geomap has two
   attributes, {\em loc} and {\em contains}, that are used to designate 
   containment. Containers provide many convenient benefits.

   %\item Can have different coordinate system for each container if
   %  needed.
   %  \begin{itemize}
   %  \item Only need to change the container position and all things
   %    that use it as a reference system change too (for example
   %    ship/character moving).
   %  \item Can have one generic floating point system and still have
   %    full accuracy at detailed levels. For example one system for
   %    the galaxies and start from 0.0,0.0,0.0 for the planets.
   %  \end{itemize}
   Since a container is independent from any other, each may have its own 
   coordinate system to designate the position of objects within it.  This 
   allows the ability to change the container's position without
   actually having to modify the position of those objects contained
   within it.
   Second, the coordinate system can be one generic floating point system, 
   while still providing accurate positioning at detailed levels.  Though each container 
   uses the same system, the depth of containment would specify the actual scale
   between two containers.  See the examples later for further explanation. 
   ~(I assume this is further explained in examples below)
   
   %\item Need to transmit only things client see and only need one
   %  mechanism for all of it.
   %  \begin{itemize}
   %    \item Don't need to show what the box contains.
   %    \item No need to show what a neighbor forest contains.
   %    \item Also solar system movements or the planet characters are
   %      playing on don't need to be visible to characters that are
   %      living it in medieval world.
   %    \end{itemize}
   %\item Can have different data at different zoom levels.
   %  \begin{itemize}
   %  \item For things that are far away send container info, but not
   %    what they contain at bottom level (like looking from a mountain
   %    to far away lands or standing near {\em big} a cliff).
   %  \item You can view overview the geomap for travel etc.. purposes
   %  \item Can make big/generic changes more easily.
   %  \end{itemize}
   Another benefit of containers is scope.  With certain actions, such as sight
   or collision detection(CD), containers provide a way to limit which objects
   need to interact.  For example, there is no need to display what items a box
   might contain or to show objects in a neighboring forest. Scope also can limit
   the detail of information in the same way.  For instance, someone standing on 
   a mountain looking out would see other mountains, forests, and perhaps cities, 
   but would not know what was contained within them.

   %\item It is a way to group things logically/geographically. This is
   %  handy if one wishes to distribute sections of the world onto
   %  different game servers and/or assemble a realm from ``canned''
   %  realms.
   Lastly, containers provide a way to logically group objects.  This
   can be a useful
   way to define sections of a game world, so in can be distributed to several game 
   servers, or they could be ``canned'' into a packaged realm.

   %\end{enumerate}

   \subsection{Container examples}
   
   Only attributes related to ``containership'' are included. Note
   that orientation is not included in these examples\ldots yet.

   \begin{Aobjectfigure}
     \begin{Amap}
       \Attr{id}{``galaxy1''}
       \Attr{name}{Milkway}
       \Attr{contains}{(``ss2'', ``ss3'', ``ss4)}
     \end{Amap}

     \begin{Amap}
       \Attr{id}{``ss2''}
       \Attr{name}{``Solar system''}
       \Attr{loc}{``galaxy1''}
       \Apos{\bn{2.6}{20}}{\bn{-1.5}{19}}{\bn{1.5}{19}}
       \Attr{contains}{(``venus5'', ``earth6'', ``mars7'', ``jupiter8'')}
     \end{Amap}

     \begin{Amap}
       \Attr{id}{``earth6''}
       \Attr{name}{``Planet Earth''}
       \Attr{loc}{``ss2''}
       \Apos{\bn{1.5}{11}}{0.0}{0.0}
       \Attr{contains}{(``dural9'', ``trans10'')}
     \end{Amap}

     \begin{Amap}
       \Attr{id}{``dural9''}
       \Attr{name}{``Dural world''}
       \Attr{loc}{``earth6''}
       \Apos{\bn{5.0}{3}}{\bn{4.5}{6}}{\bn{4.5}{6}}
       \Attr{contains}{(``cambria11'', ``moraf12'')}
     \end{Amap}

     \begin{Amap}
       \Attr{id}{``cambria11''}
       \Attr{name}{``Cambria''}
       \Attr{loc}{``dural9''}
       \Apos{\bn{8.0}{4}}{\bn{-1.6}{4}}{200.0}
       \Attr{contains}{(``agrilan13'', ``summerset14'')}
     \end{Amap}

     \caption{Containers examples, part 1}
   \end{Aobjectfigure}

   \begin{Aobjectfigure}
     \begin{Amap}
       \Attr{id}{``agrilan13''}
       \Attr{name}{``Agrilan''}
       \Attr{loc}{``cambria11''}
       \Apos{\bn{-3.2}{4}}{\bn{2.0}{4}}{100.0}
       \Attr{contains}{(``forest123'', ``marketplace321'')}
     \end{Amap}

     \begin{Amap}
       \Attr{id}{``forest123''}
       \Attr{name}{`Haretone forest''}
       \Attr{loc}{``agrilan13''}
       \Apos{-100.0}{-200.0}{10.0}
       \Attr{contains}{(``oak1234'', ``34555'', ``34556'', ``34557'')}
     \end{Amap}

     \OakTree

     \begin{Amap}
       \Attr{id}{``human234''}
       \Attr{name}{``Twaine Danylsen Junior''}
       \Apos{10.0}{3.5}{30.0}
       \Attr{loc}{``oak1234''}
       \Attr{contains}{(``backpack6788'')}
     \end{Amap}

     \begin{Amap}
       \Attr{id}{``backpack6788''}
       \Apos{0.0}{0.1}{1.0}
       \Attr{loc}{``human234''}
       \Attr{contains}{(``apple11111'')}
     \end{Amap}

     \begin{Amap}
       \Attr{id}{``apple11111''}
       \Attr{loc}{``backpack6788''}
     \end{Amap}

     \caption{Containers examples, part 2}
   \end{Aobjectfigure}
   
   \begin{itemize}
   \item The galaxy inherits default $pos=\floatthree{0.0}{0.0}{0.0}$
     from ``game\_entity'' class definition.
   \item Before space travel is implemented, server will omit $loc$
     and $pos$ attributes from ``dural9'' object and thus `pretend'
     it's topmost level object for clients. 
   \item When looking from far away at the Earth, the $contains$ attribute
     will be shown as empty and only filled when the spaceship is
     near enough to see ``dural9''. Similarly if Shewla Danylsen is
     too far from ``oak123'' tree, she won't see her son. Also unless
     ``backpack6788'' is open, the backpack's content is not visible.
   \item Note that ``apple11111'' does not have $pos$ attribute. It's
     position inside ``backpack6788'' is undetermined.
   \item In the example,  ``backpack6788'' is defined using numerical
     coordinates. Another possibility might be to set it's $loc$
     attribute to ``human234\#back''. Another possibility might be to
     use something like $symbolic\_pos$ with ``back'' as value.
   \end{itemize}

   \newpage

   \subsection{Example on how to calculate ``absolute'' position}
   
   This example uses python syntax.

   \begin{verbatim}
def get_xyz(self):
    """get location: if relative to location (container),
       add our position to actual container location"""
    if self.loc:
        return self.loc.get_xyz() + self.pos
    else:
        return self.pos
   \end{verbatim}
   
   Your actual implementation can certainly optimize beyond the above
   code. It can, for example, also store an $absolutePosition$
   attribute. You'd calculate that for every object and when the object
   moves, update both $pos$ and $absolutePosition$ attributes. When
   object changes containers, you'd calculate it's new $absolutePosition$
   using the formula $loc.absolutePosition + pos$.

   \subsection{Alternative: containers as separate objects}
   
   $loc$, $pos$ and $contains$ attributes could be made into separate
   objects. This object would also have separate class hierarchy.
   

   Discussion with bryce about containers:

   ``In the real world, every item contained within another item has a
   specific position and identity.  However, in some situations it is
   immaterial where the item is, exactly, and in others we don't care
   too much about the specificality of the item.  Knowing that we don't
   care, we can then throw that information away and compress our data
   down by a great deal.  

   For example, consider a bag full of apples, pears, and bananas.
   Certainly in real life, each of these items has a specific position
   within the bag, however within a game server if we had to go to the 
   work of calculating exact positions of each piece of fruit, taking
   into account physical shape, smooshiness, etc. it'd be a waste of
   cycles.  Instead, all we really, truly need to know is that within
   the bag there are 4 apples, 2 pears, and 7 bananas.  We don't care
   about the position.  This not only saves us the processing work of
   positioning the items, but also allows us to skip keeping track of
   the positions of the items.

   Alternatively, consider an orchard's 300m x 300m field of apple
   trees.  The field consists of rows 3 meters apart, with each tree
   within a row spaced 3 meter apart.  In a generic container, we'd need
   to describe the area by sending data on 10,000 entities to the
   client, but because every tree is identical, and because they're
   spaced in a deterministic grid, this is extremely wasteful.  Instead,
   what we want to do is tell the client, ``Put 10,000 Apple Trees in a
   grid according to this formula...''  

   As you can surmise, there are probably other different styles of
   containers beyond 'bags' and 'fields', so we generalize the notion of
   a ``container'', separately from an ``entity''.  An entity could
   contain other entities in different ways, depending on how we want to
   store things within it.  For example, consider if we cut down all the
   apple trees and pile them up in a jumble along with bits of broken
   stone and some dead bodies.  In this case we may want to treat the
   field just like a bag - a random collection of specific items.  On
   the other hand, if we cleared the field and two groups of brigands
   faced off across it, engaged in battle, we'd suddenly care about both
   the exact positions and exact natures of everything on the field.

   Thus, we want to allow a given entity to be able to change its type
   of container, as needed.  This is modelled by saying that ``an entity
   has a container'' instead of ``entity is a container''.   

   Here are some kinds of containers:

   Map - Items have a specific identity and a specific position.

   Pile - Items do not have specific positions and are assumed to be
   randomly distributed within the container.  The quantity of items of
   each specific type is given.

   List - Items are ordered and have a specific identity, but are not
   geospacially positioned.  This can be thought of as a queue or
   stack.  

   Set - Rather than using 3D coordinates, this uses abstract named
   ``locations'' within the container.  For example, within a spacecraft
   we could identify crew positions as ``Pilot'', ``Gunner 1'', ``Gunner
   2'', ``Engineer'', etc.  In such circumstances it is assumed that the
   ``role'' is much more important for the game to know than the
   coordinate location.  In a ship-to-ship space combat game,
   coordinates within a ship may be irrelevant (unless it's being
   boarded) than it is to know who's in which chair.   In a roleplaying
   game, this style of container would be used for a character's
   possessions - he'd have named locations ``neck'', ``torso'', ``left
   foot'', and ``top of head''.  An article of clothing (or weapon or
   armor) could be placed in each of these named locations.


   In container info inside object model above might be modeled by
   having special object and/or special attributes.

   \newpage

   \section{Cartography/Geometry at different zoom levels}
   
   \begin{itemize}
   \item Object can be a point. Usually characters, trees, etc\ldots
     are like this.
   \item Object can be a line. Usually roads, fences, etc\ldots are
     like this.
   \item Object can be an area. Usually fields, forests, etc\ldots are
     like this.
   \item Object can be a volume. Usually caves, houses, etc\ldots
     are like this.
   \end{itemize}
   
   One object can be different kind of geometry type at different zoom
   levels. For example you are far away from some star. You see a star
   as point object. When you get near the star you see it as volume
   object. Then get even nearer and see river as line object in planet
   that orbits the star. You ascend toward the river. You see the big
   river as an area object. You dive into it and see it as volume
   object. \footnote{If server is using 2 coordinates and client is
     using 3 coordinates, then client should add/remove element
     as needed: $Server:(x,y)$ becomes $Client:(x,y,0.0)$,
     $Client:(x,y,z)$ becomes $Server:(x,y)$.  Similarly also if
     server has 4 or more coordinates and client is using 3
     coordinates, client should add/remove elements as needed.}
   
   More detailed object(s) should be referred in $contains$ attribute.

   \subsection{Point (0D) objects}
   
   This and following cartography subsections are based largely on
   discussion with James Turner, Anubis, DragonM, tbp, bear, malcolm
   and Jorrit (in order of time).

   Point objects have only position.

   \begin{Aobjectfigure}
     \begin{Amap}
       \Attr{id}{``human234''}
       \Attr{name}{``Twaine Danylsen Junior''}
       \Apos{10.0}{3.5}{30.0}
       \Attr{loc}{``oak1234''}
       \Attr{contains}{(``backpack6788'')}
     \end{Amap}
     \caption{Human point object}
   \end{Aobjectfigure}
   
   Also there can be ``point data'' objects which have coordinates for
   many positions. These objects are referred by line, area and volume
   objects. \footnote{Using $p$ in point data objects instead of
     $points$ because it will be referred more than any other
     attribute: in binary2 encoding length doesn't matter but for
     every other encoding it does matter.}


   \begin{Aobjectfigure}
     \begin{Amap}
       \Attr{id}{``center1''}
       \Attr{p}{$((10.0, 2.0, 2.0), (5.0, -1.0, 1.0), (0.0, 0.0, 0.0))$}
     \end{Amap}
     \begin{Amap}
       \Attr{id}{``outline1''}
       \Attr{p}{$((10.0, 2.5, 2.0), (10.0, 1.5, 2.0),$}
       \Attr{}{$\ (7.0, -0.7, 1.6), (5.0, -1.5, 1.0), (5.0, -0.5, 1.0))$}
     \end{Amap}
     \begin{Amap}
       \Attr{id}{``bottom1''}
       \Attr{p}{$((10.0, 2.0, 1.8), (5.0, -1.0, 0.8))$}
     \end{Amap}
     \caption{Point data objects for river, all points could be in one
       object if wanted}
   \end{Aobjectfigure}
   
   \subsection{Line (1D) objects}
   
   Line objects refer to point objects in polyline attribute. Format
   is $object\_id.attribute\_name.index$. Even if polyline is made
   loopy, it does not cover area: center is empty unless there is area
   object.

   \begin{Aobjectfigure}
     \begin{Amap}
       \Attr{id}{``river9''}
       \Attr{parents}{(``river'')}
       \Attr{polyline}{(``center1.p.0'', ``center1.p.1'',
         ``center1.p.2'')}
       \Attr{contains}{(``river\_area4'')}
     \end{Amap}
     \caption{Line object (river): points attribute could be included
       here also, reference then would be ``river9.points.0'', \ldots}
   \end{Aobjectfigure}

   \begin{Aobjectfigure}
     \begin{Amap}
       \Attr{id}{``r5''}
       \Attr{parents}{(``rope'')}
       \Attr{polyline}{(``joe123.pos'', ``mic8.pos'')}
     \end{Amap}
     \caption{Rope: Joe is holding other end and Michael is holding
       other end}
   \end{Aobjectfigure}
   

   \newpage

   \subsection{Area (2D) objects}
   
   Area objects are triangles meshes. They refer to point objects in
   area attribute. Each triangle is defined by 3 ids to points. Every
   triangle should share at least one edge with another triangle and
   you can get into any triangle from any other triangle by traveling
   from triangle to another. If you need areas that consists of
   separate pieces, make higher level object and ``contain'' area
   objects in it for that. \footnote{Even if area is made loopy so it
     seems like it covers volume, it doesn't: there is no inside and
     outside and both sides are empty unless there is real volume
     object.}

   \begin{verbatim}
until figure is made, here is ascii version...
        ____
       /    0
      /  ___1
     4 _2
     3/

   \end{verbatim}
%0-> 1,2,4 b0,b1
%1-> 0,2�b0
%2-> 0,1,3,4 b0,b1
%3-> 2,4 b1
%4-> 0,2,3 b0,b1
%b0-> 0,1,2,4 b1
%b1-> 0,2,3,4 b0
   \begin{Aobjectfigure}
     \begin{Amap}
       \Attr{id}{``river\_area4''}
       \Attr{parents}{(``river\_area'')}
       \Attr{area}{((``outline1.p.0'', ``outline1.p.1'', ``outline1.p.2''),}
       \Attr{}{(``outline1.p.2'', ``outline1.p.3'', ``outline1.p.4''),}
       \Attr{}{(``outline1.p.0'', ``outline1.p.2'', ``outline1.p.4''))}
       \Attr{contains}{(``river\_volume10'')}
     \end{Amap}
     \caption{River area object}
   \end{Aobjectfigure}

   \newpage
   
   Area objects can also have holes in them. Holes are define by
   missing triangles inside area. \footnote{In binary2 river and lake
     example objects can be encoded as series of int16 numbers or even
     as series of int8 numbers}

   \begin{verbatim}
until figure is made, here is ascii version...
 0    1
   45 
   67
 2    3
   \end{verbatim}

   \begin{Aobjectfigure}
     \begin{Amap}
       \Attr{id}{``lake\_area5''}
       \Attr{parents}{(``lake\_area'')}
       \Attr{area}{((``pd.p.0'', ``pd.p.1'', ``pd.p.4''),}
       \Attr{}{(``pd.p.1'', ``pd.p.5'', ``pd.p.4''),}
       \Attr{}{(``pd.p.1'', ``pd.p.3'', ``pd.p.5''),}
       \Attr{}{(``pd.p.3'', ``pd.p.7'', ``pd.p.5''),}
       \Attr{}{(``pd.p.3'', ``pd.p.6'', ``pd.p.7''),}
       \Attr{}{(``pd.p.3'', ``pd.p.2'', ``pd.p.6''),}
       \Attr{}{(``pd.p.0'', ``pd.p.6'', ``pd.p.2''),}
       \Attr{}{(``pd.p.0'', ``pd.p.4'', ``pd.p.6''))}
       \Attr{contains}{(``river\_volume14'')}
     \end{Amap}
     \caption{River area object with island ``holes''}
   \end{Aobjectfigure}
   
   Also could make holes by using ``negative'' area: island is
   contained by lake and removes water from it's area.

   \subsection{Volume (3D) objects}
   
   Volume objects consist of one or more tetrahedrons. They refer to
   point objects in volume attribute. Each tetrahedrons is defined by
   4 ids to points. Every tetrahedrons should share at least one face
   with another tetrahedrons and you can get into any tetrahedrons
   from any other tetrahedrons by traveling from tetrahedrons to
   another. If you need volumes that consists of separate pieces, make
   higher level object and ``contain'' volume objects in it for that.

   \begin{Aobjectfigure}
     \begin{Amap}
       \Attr{id}{``river\_volume10''}
       \Attr{parents}{(``river\_volume'')}
       \Attr{volume}{((``outline1.p.0'', ``outline1.p.1'', ``bottom1.p.0'', ``outline1.p.2''),}
       \Attr{}{(``outline1.p.2'', ``outline1.p.3'', ``bottom1.p.1'', ``outline1.p.4''),}
       \Attr{}{(``outline1.p.0'', ``outline1.p.2'', ``bottom1.p.1'', ``outline1.p.4''),}
       \Attr{}{(``outline1.p.2'', ``bottom1.p.0'', ``bottom1.p.1'', ``outline1.p.4''),}
       \Attr{}{(``outline1.p.0'', ``bottom1.p.0'', ``bottom1.p.1'', ``outline1.p.2''))}
     \end{Amap}
     \caption{Volume river object: surface and bottom}
   \end{Aobjectfigure}
   
   Volume objects can have 1 or more holes or empty spaces by missing
   tetrahedrons somewhere. Alternatively they can contain ``negative
   space'': for example mountain contains cave or house contains room.

   \subsection{Higher (4D or more) dimensional objects}

   Higher dimensional object is defined using collection of primitive
   objects which are defined using N+1 points and with N+1 N-1
   dimensional limiting objects similarly than 2D and 3D is done.

   \subsection{Connecting neighbour objects}
   
   For topological consistency neigbour objects should be connected to
   each other via common points/lines/triangles meshes/etc... More
   about this later.

   \section{Hyperlinks}
   
   URI (ids in Atlas) consists of form: atlas://server.org/acorn/root\\
   Protocol is ``atlas://''.\\
   Server is ``server.org''.\\
   World id is ``acorn''.\\
   Root object (=id) is ``root''.
   
   Maybe ``world id'' and ``id'' part should be separated by something
   different than just '/'?

   \section{Media}
   
   Media is completely separate. Objects in world server don't know
   anything about media objects. World server only gives list of
   recommended media server(s) but client can override selection.

   So how does client know which media to use for objects?
   
   It uses object id, parents and other attributes to lookup for it at
   media server. First client searches using object id, then using
   parents ids, then grandparents ids, etc.. until it finds match.
   Searches are cached on client and probably makes good sense to
   cache list of ids media server offers.

   \subsection{Example lookup}

   Client sees character named ``Halek''.

   \begin{Aobjectfigure}
     \begin{Amap}
       \Attr{id}{``ch98''}
       \Attr{name}{``Halek''}
       \Attr{parents}{(``elf'')}
     \end{Amap}
     \begin{Amap}
       \Attr{id}{``elf''}
       \Attr{parents}{(``humanoid'')}
     \end{Amap}
     \caption{``Halek'' character and ``elf'' definition}
   \end{Aobjectfigure}
   
   First client queries media server for info about ``ch98''. If no
   info about that exist, then it queries about ``elf''. If still no
   info, then for ``humanoid''. Client can fetch all info objects from
   media server beforehand, so querying in practice can be just cache
   lookup. Lets assume client finds media info object for ``elf''.

   \begin{Aobjectfigure}
     \begin{Amap}
       \Attr{id}{``elf''}
       \Attr{parents}{(``humanoid'')}
       \Attr{media\_groups}{(``media\_2D\_iso'', ``media\_3D'', ``media\_text'')}
       \Attr{media\_2D\_iso}{(``elf\_2D\_iso\_male'', ``elf\_2D\_iso\_female'')}
       \Attr{media\_3D}{(``elf\_3D\_male'', ``elf\_3D\_female'')}
       \Attr{media\_text}{(``elf\_text\_male'')}
     \end{Amap}

     \begin{Amap}
       \Attr{id}{``elf''}
       \Attr{parents}{(``humanoid'')}
       \Attr{media\_groups}{
         \begin{Amap}
           \Attr{media\_2D\_iso}{(``elf\_2D\_iso\_male'', ``elf\_2D\_iso\_female'')}
           \Attr{media\_3D}{(``elf\_3D\_male'', ``elf\_3D\_female'')}
           \Attr{media\_text}{(``elf\_text\_male'')}
         \end{Amap}
         }
     \end{Amap}
     \caption{Top level media info object for ``elf'': 2 alternative versions}
   \end{Aobjectfigure}
   
   \begin{Aobjectfigure}
     \begin{Amap}
       \Attr{id}{``elf\_3D\_female''}
       \Attr{parents}{(``elf\_3D\_humanoid'')}
       \Attr{sex}{``female''}
       \Attr{media\_url}{``http://purple.worldforge.org/wf/cvs/media-3d/character/elf\_female.3ds''}
     \end{Amap}
     \caption{Media info object for 3D female elf}
   \end{Aobjectfigure}

   \newpage

   \subsection{Filtering}
   
   Media info objects can contain also other attributes to use for
   filtering than ids. For example there might be different media
   available for different sized objects.

   

%   \subsection{2D/3D/text, etc\ldots media}

%   \section{MIM and Semantic GeoMaps}

\end{document}
