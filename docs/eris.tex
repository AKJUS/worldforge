%% ================================================================================
%% This LaTeX file was created by AbiWord.                                         
%% AbiWord is a free, Open Source word processor.                                  
%% You may obtain more information about AbiWord at www.abisource.com              
%% ================================================================================

%\documentclass[12pt]{article}
%\usepackage[T1]{fontenc}
%\usepackage{calc}
%\usepackage{setspace}
%\usepackage{multicol}	% TODO: I don't need this package if the document is a single column one.
%\usepackage[normalem]{ulem}	% TODO: Package is only needed if you have underline/strikeout.
%\usepackage{color}	% TODO: Package is only needed if you have color.

\def\myback{\ensuremath{\backslash}}

%\begin{document}

\documentclass{article}

\author{James Turner}

\begin{document}



\section{Introduction}



\subsection{Prerequisites}

Eris relies heavily on a number of concepts which you should be familiar with
(to some degree). Firstly, C++ in general, including the STL (notably containers and
iterators), as well as templates,  namespace and exceptions. Secondly, Eris
makes extensive use of libSigC++ to implement versatile callbacks between objects.
Familiarity with the concepts of signals/slots (as found in Qt, Gtk+ and others) is
invaluable. Basic introductions can be found in many places.



Some familiarity with Atlas is assumed. While Eris does encapsulate a good deal of Atlas
from client authors, achieving good integration and extending the behavior
of the system will require dealing with Atlas objects. There are two aspects, either of
which will prove useful; firstly familiarity with the Atlas spec, and the precise meanings
of terms such as codec, entity and operation, and secondly a working knowledge of AtlasC++ and how it implements the spec.



\section{Working with Eris}



This section intends to cover the essential points of writing a client using Eris. Establishing
all the issues which need to be discussed here is a large task, so please don't consider
this to be complete, at least for a while. In addition to this document, possibly
the best material available is the source code to other Eris clients, notably SilenceC++,
which doubles as a testharness and trivial example of various aspects of Eris functionality.



To avoid duplication with the design section, frequent reference will be made to various
classes without providing much background information; refer to the appropriate design section
for that information.



\subsection{Getting Started}

The first task is to get Eris and it's dependencies compiled and installed on your system.
This document will assume a Unixstyle build system for the moment, based around
automake and autoconf.



The most basic Eris program looks like this:

{\tt



\#include <Eris/Connection.h>

\#include <Eris/PollDefault.h>



void main(int argc, char* argv[])

\{

    Eris::Connection myCon("test\myback\_client\myback\_id", false);

    

    myCon.connect("server.worldforge.org", 6767);

    

    while (1) \{

	Eris::PollDefault::poll();

    \}

\}

}



There are a few things to note about this example:

\begin{itemize}

\item Every Eris function and class is declared in the Eris namespace. You can choose
to insert 'using namespace Eris;' declarations in your source files, or prefix all
occurrence with 'Eris::' as clarity and brevity permit.

\item All the Eris headers are grouped inside an 'Eris' directory 

\end{itemize}



The code above creates a connection object (which there should be only one of for the entire
program, at the moment), and then enters a loop (which would be your mainloop in a real
application) which calls a poll function. Since Eris is deliberately not multithreaded,
and does not make assumptions about the environment it runs in (for example, setting up
alarms or signals), it requires you to manually poll it for data reasonably frequently.
However, a standard implementation is provided for convenience (PollDefault) which
simply calls the POSIX {\tt select()} function with a zerolength timeout.



The program above, when run, will open a connection to the specified server, perform
Atlas negotiation to determine a codec, and then start synchronizing the local and remote
Atlas typeinfo databases. However, it has a major flaw in that your client code has no
way of monitoring what the Eris components are doing.



\subsection{Signals and Slots}



As noted above, Eris is not threaded, and does not require threading in clients. Instead,
it is fully asynchrous, and uses callbacks to notify the client code about important
events it might wish to respond too. In order to make the example above more useful,
we need to attach some code to those callbacks. In Eris, callbacks are implemented as
SigC++ signals.



\subsection{Callbacks in Practice}



{\tt



\#include <Eris/Connection.h>

\#include <Eris/PollDefault.h>



void main(int argc, char* argv[])

\{

    Eris::Connection myCon("test\myback\_client\myback\_id", false);

    

    {\b myCon.Failure.connect(SigC::slot(\&failureCallback)); }

    {\b myCon.Connected.connect(SigC::slot(\&connectedCallback)); }

    myCon.connect("server.worldforge.org", 6767);

    

    while (1) \{

	Eris::PollDefault::poll();

    \}

\}




void connectedCallback()

\{

    std::cout << "connected to the server" << std::endl;

\}



void failureCallback( const std::string \&msg)

\{

    std::cerr << "Eris reported error: " << msg << std::endl;

\}

}



The additional code here comprises two functions (which could equally be methods
or any other object that can become a SigC {\tt slot}), and some logic to attach these
functions to the 'Failure' and 'Connection' signals that Eris::Connection defines.
The implementation of the callbacks is entirely up to you; for example the
Connected callback (and the matching Disconnected signal) could be used to toggle
an icon in your client's status bar. Similarly the Failure callback could be used to
pop up an alert dialog, or write to an ingame console.



\section{Design}



\subsection{Overview}



The Eris classes and objects can be grouped into four basic clusters

\begin{itemize}

\item The lowlevel network types, notably Connection, TypeInfo, the
dispatchers and the utility classes such as Timeout.

\item The outofgame session objects, principally Room and Lobby, plus
a small helper class, Person.

\item The ingame session objects, most obviously World and Entity. Associated
(but less visible) classes are the Factories and the InvisibleEntityCache.

\item The metaserver object, essentially Meta, MetaQuery and ServerInfo

\end{itemize}



In addition there is a helper class, 'Player', which does not fit neatly into these
categories, but exists to encapsulate the notion of a playing user, with the
expectation that multiple players may be supported in the future.



\subsection{Connection}



Eris::Connection encapsulates the network stream (as provided by the skstream library),
as well as the Atlas codecs and decoders. In addition, it manages Atlas negotiation just
after the steam is successfully opened. Connection is critical because it is the root of the
Dispatcher hierarchy which routes incoming messages to the correct object.



Immediately after a successful Atlas negotiation completes, Connection initiates initial
synchronization of the Atlas type trees stored locally and on the server. This process can
take several seconds depending on what types are available locally, the size of the server's
type tree, and so on. Having accurate dynamic type information about Atlas objects is
critical to enable intelligent process of operations and entities by Eris and the clients,
as will be explained below.  



\subsection{Dispatchers}



Dispatchers are objects that filter and route Atlas::Message::Objects received by the Connection
from the network. Branch dispatchers typically test some attribute of the message, such as it's
sender, type, contents, etc against a stored value. If the test is passed, the message is
propagated to all of the dispatcher's children. If the test fails, propagation is stopped.



A few branch dispatchers differ slightly in this behavior, in that they examine an
attribute of the message, and route to exactly one of their children. This behavior is exhibited
by the ClassDispatcher, which routes based on the Atlas inheritance of messages. This
behavior allows child dispatchers to handle derived object types (for example SIGHT operations, which
inherit from PERCEPTION and hence INFO) without another child registered for INFO operations being invoked.



The leaf Dispatchers are much simpler, simply forwarding received messages
on to a destination. The most common example of this is the Signal dispatcher, which
is a template class. It first converts the Atlas message into an Atlas object (which
is the specialization type of the dispatcher), and the invokes a SigC::Signal passing
the Atlas object as it's argument.



Dispatchers can modify messages the propagate to their children; the most common
example of this being the Encap (encapsulation) dispatcher, which propagates the object
stored in some argument of an operation to it's children. The 'container' objects is pushed
down a stack, so it is still available. In the case of certain dispatches
(such as SIGHT(CREATE(...)) ), this process occurs multiple times. The motivation for retaining
the outer objects (the SIGHT in the example above) is that often critical information is
contained. To take advantage of this, variants of the Signal dispatcher exist which
decode and pass the top two items of the stack (and more is possible).



One final point is that dispatchers have a unique name in their parent context, and
this can be used to address a given dispatcher instance using a path specification. Notably,
Connection features a {\tt getDispatcherByPath} method (which may return NULL). In order
to simplify the addressing of dispatchers, there is support for anonymous nodes, which
are generally handled transparently by the system. Dispatchers are reference counted,
so once added to the tree they should manage themselves. This referencecounting scheme
also permits a dispatcher to have multiple parents (which is used in practice) and even
loops (which has not been tried yet).



The dispatcher system is the most complex part of Eris, and had taken considerable
evolution to reach it's current form. The price of moderate complexity in design is balanced
by a small amount of code (most of the dispatchers are two methods, a constructor and the
dispatch' implementation), and extreme dynamic flexibility. This makes dealing with
new Atlas types and changes to the protocol specification trivial.



\subsection{The InGame Classes}



The most important ingame classes are World and Entity. World is a slight misnomer,
in that it represents the ingame session, which is quite different to the server's
notion of the World. Nevertheless, it acts as a container for every Entity visible to
the client, so the name will probably remain (it was inherited from UClient). World
acts as a registrar for Factories, which will be discussed further below, and provides
global callbacks to monitor Entity lifetime (creation/deletion and appearance/disappearance).
These callbacks permit clients to interface with Eris in various ways. The most common
is to attach every Entity to a global object via it's callbacks (Moved, Changes and so on).
The other is to create a peer/proxy instance in response to EntityCreated or EntityAppeared,
and have that peer class be notified via the callbacks.



This allows flexible integration with other code. For example, if working with a scene
graph library, this permits a visualization node to be created when an entity appears.
The 'Moved' callback can be linked to a method on the scene graph node which updates
the transform used in rendering.



By using callbacks, it is no longer necessary to achieve all the integration by
subclassing (often involving multiple inheritance). To support much closer integration,
Eris supports subclassing of Entity via Factories. These are objects registered with
the World, which are (in order, based on an integer priority) passed each new Atlas
object representing an Entity. If they choose to 'accept' an entity, they should create
whatever instance type they choose. In combination with the TypeInfo system, this
permits a client to create instances of MyClientPerson for all entities inheriting from
'person', while creating instances of MyClientTree for every entity inheriting from 'tree'.
Of course clients may choose to use one class for every entity.



If you override certain methods in your subclass (notably those used to track the state
of Atlas attributes) and fail to call the original code you may cause undefined behavior
by Eris : please take great care if doing so, and consider whether such changes are
necessary. Subclassing is intended to permit complex integration and processing : you
shouldn't need it to get a basic client working. 



\section{The Road Ahead}







\section{History} 



The initial work on Eris was done during June 2001, with the goal of potentially demonstrating
a 3D client at LinuxTag (in July!). Given an existing rendering system, what was
required was the Atlasspeaking backend, and code to track entities outside of Atlas
operations, so that a render could query position, velocity and other attributes independently
of the network. At the time of writing, the only WorldForge platform capable of supporting a
working game was Acorn 0.4, i.e CyphesisC++ and UClient. Thus, I began by studying the UClient
code base, locating all the classes which dealt with Atlas and the network, and their
interface to the renderer and sprite systems.



This lead to the initial Connection, Player, Entity and World classes (though in much more
basic form). Sometime during this period I also despaired of the Atlas::Objects::Decoder and
came up with the initial version of the Dispatcher system.



Going through the UClient code revealed many areas where the Acorn 0.3/0.4 platform contained oddities
or missing functionality. From June through till November there is a steady stream of
incremental adjustments to the spec based on the hindsight available. After LinuxTag, the
outofgame code evolves into reliable operation with
both cyphesisC++ and Stage. This process continually reveals areas of inconsistency
between the two servers which are gradually addresses and resolved over the next few months.



...




\end{document}
